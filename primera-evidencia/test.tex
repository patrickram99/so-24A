%package list
\documentclass{article}
\usepackage[top=3cm, bottom=3cm, outer=3cm, inner=3cm]{geometry}
\usepackage{graphicx}
\usepackage{url}
%\usepackage{cite}
\usepackage{hyperref}
\usepackage{array}
%\usepackage{multicol}
\newcolumntype{x}[1]{>{\centering\arraybackslash\hspace{0pt}}p{#1}}
\usepackage{natbib}
\usepackage{pdfpages}
\usepackage{multirow}
\usepackage{multirow}
\usepackage[T1]{fontenc}
\usepackage{imakeidx}
\usepackage[normalem]{ulem}
\useunder{\uline}{\ul}{}

% codigo fuente
\usepackage{listings}
\usepackage{color, colortbl}
\definecolor{dkgreen}{rgb}{0,0.6,0}
\definecolor{gray}{rgb}{0.5,0.5,0.5}
\definecolor{mauve}{rgb}{0.58,0,0.82}
\definecolor{codebackground}{rgb}{0.95, 0.95, 0.92}
\definecolor{tablebackground}{rgb}{0.0, 0.45, 0.63}
\lstset{frame=tb,
	language=bash,
	aboveskip=3mm,
	belowskip=3mm,
	showstringspaces=false,
	columns=flexible,
	basicstyle={\small\ttfamily},
	numbers=none,
	numberstyle=\tiny\color{gray},
	keywordstyle=\color{blue},
	commentstyle=\color{dkgreen},
	stringstyle=\color{mauve},
	breaklines=true,
	breakatwhitespace=true,
	tabsize=3,
	backgroundcolor= \color{codebackground},
}

%%%%%%%%%%%%%%%%%%%%%%%%%%%%%%%%%%%%%%%%%%%%%%%%%%%%%%%%%%%%%%%%%%%%%%%%%%%%
%%%%%%%%%%%%%%%%%%%%%%%%%%%%%%%%%%%%%%%%%%%%%%%%%%%%%%%%%%%%%%%%%%%%%%%%%%%%
\newcommand{\csemail}{vmachacaa@ulasalle.edu.pe}
\newcommand{\csdocente}{MSc. Maribel Molina Barriga}
\newcommand{\cscurso}{Sistemas Operativos}
\newcommand{\csuniversidad}{Universidad La Salle}
\newcommand{\csescuela}{Escuela Profesional de Ingeniería de Software}
\newcommand{\cspracnr}{01}
\newcommand{\cstema}{Instalación Debian}
%%%%%%%%%%%%%%%%%%%%%%%%%%%%%%%%%%%%%%%%%%%%%%%%%%%%%%%%%%%%%%%%%%%%%%%%%%%%
%%%%%%%%%%%%%%%%%%%%%%%%%%%%%%%%%%%%%%%%%%%%%%%%%%%%%%%%%%%%%%%%%%%%%%%%%%%%


\usepackage[english,spanish]{babel}
\usepackage[utf8]{inputenc}
\AtBeginDocument{\selectlanguage{spanish}}
\renewcommand{\figurename}{Figura}
\renewcommand{\refname}{Referencias}
\renewcommand{\tablename}{Tabla} %esto no funciona cuando se usa babel
\AtBeginDocument{%
	\renewcommand\tablename{Tabla}
}

\usepackage{fancyhdr}
\pagestyle{fancy}
\fancyhf{}
\setlength{\headheight}{30pt}
\renewcommand{\headrulewidth}{1pt}
\renewcommand{\footrulewidth}{1pt}
\fancyhead[L]{\raisebox{-0.2\height}{\includegraphics[width=3cm]{logo_ulasalle (1).png}}}
\fancyhead[C]{}
\fancyhead[R]{\fontsize{7}{7}\selectfont	\csuniversidad \\ \csescuela \\ \textbf{\cscurso} }
\fancyfoot[L]{}
\fancyfoot[C]{Sistemas Operativos}
\fancyfoot[R]{Página \thepage}



\begin{document}

	\vspace*{10px}
	
	\begin{center}	
		\fontsize{17}{17} \textbf{ Práctica \cspracnr}
	\end{center}
	%\centerline{\textbf{\underline{\Large Título: Informe de revisión del estado del arte}}}
	%\vspace*{0.5cm}
	

\renewcommand{\arraystretch}{1.5}
\begin{table}[h]
	\begin{tabular}{|x{4.7cm}|x{4.8cm}|x{4.8cm}|}
		\hline 
		\textbf{DOCENTE} & \textbf{CARRERA}  & \textbf{CURSO}   \\
		\hline 
		\csdocente & \csescuela & \cscurso    \\
		\hline 
	\end{tabular}
\end{table}	

\begin{table}[h]
	\begin{tabular}{|x{4.7cm}|x{4.8cm}|x{4.8cm}|}
		\hline 
		\textbf{GRUPO} & \textbf{TEMA}  & \textbf{DURACIÓN}   \\
		\hline 
		\ 6 & \cstema & 5 horas   \\
		\hline 
	\end{tabular}
\end{table}
\renewcommand{\arraystretch}{1} % Reset the padding to the default value

	\section*{Integrantes}

	 	\begin{itemize}
                    \item José Carlos Machaca Vera
	 		\item Jhosep Alonso Mollapaza Morocco
	 		\item Patrick Andres Ramirez Santos
	 \end{itemize}
 
	\tableofcontents


	

\newpage

\section{Introduccion a Swift Script v. 1989.0}
    \subsection{Justificacion}
        Este documento presenta la creación de un lenguaje de programación inspirado en la discografía de Taylor Swift. La música de Taylor Swift es conocida por su narrativa emocional y lírica, lo que proporciona una base rica para la construcción de un lenguaje de programación. Este lenguaje busca proporcionar una forma novedosa y atractiva de aprender los conceptos de programación y compilación, al tiempo que se explora la intersección entre la música y la ingenieria de software.

    \subsection{Objetivos}
        El objetivo principal de este proyecto es diseñar e implementar un lenguaje de programación basado en la discografía de Taylor Swift. Los objetivos específicos son los siguientes:
        \begin{itemize}
            \item Desarrollar una especificación léxica y sintáctica para el lenguaje.
            \item Implementar un compilador que pueda traducir programas escritos en este lenguaje a un lenguaje de programación de alto nivel.
            \item Proporcionar ejemplos de código y documentación detallada para ayudar a los usuarios a aprender y utilizar este lenguaje.
        \end{itemize}

        \section{Propuesta}
            \subsection{Especificación léxica}
                \subsubsection{Definición de los comentarios}
                \begin{lstlisting}[caption={Comentarios}][H]
                shake Este seria un comentario de una linea
                shakeitoff
                    Este seria
                    Un comentario
                    multilinea
                shakeitoff
                \end{lstlisting}
                
                \subsubsection{Definición de los identificadores}
                El lenguage utiliza la palabra enchanted para definir una variable y los tipos se expresan al final como en Typescript
                \begin{lstlisting}[caption={Identificadores}][H]
                enchanted variable_bool : meetyou
                enchanted variable_string : wonderstruck
                enchanted variable_double = 15 : thpage 
                \end{lstlisting}
                
                \subsubsection{Definición de las palabras clave}
                e dentro de un loop representa el iterador y ee el break para for
                \begin{lstlisting}[caption={For loop}][H]
                me 1 e 12 {
                    speaknow(e) 
                    ee
                } 
                \end{lstlisting}

                oohooh sirve como break, al igual que en el for es combinar el inicio del bucle 2 veces
                \begin{lstlisting}[caption={While loop}][H]
                ooh variable > 10 {
                    speaknow("Hiii")
                    oohooh
                }                 
                \end{lstlisting}

                Se utilizan las eras para preguntar por condicionales, si estas en tu Lover era es el inicio de if, sino puede ser Red era y el ultimo recurso es Reputation era.

                \begin{lstlisting}[caption={If-Else-Elif}][H]
                loverera condition_if {
                    speaknow("hiiii")
                } redera elif_condition {
                    speaknow("hiiii")
                } repera {
                    speaknow("Else condition")
                }
                \end{lstlisting}

                Se utiliza la referencia a Speak Now para imprimir en consola y toma como argumento un tipo wonderstruck (string)

                \begin{lstlisting}[caption={Imprimir en consola}][H]
                enchanted variable_string = "Hola Mundo" : wonderstruck
                speaknow("hiiii")
                speaknow(variable_string)
                \end{lstlisting}

                \begin{lstlisting}[caption={Definir funcion}][H]
                isme nombre_funcion (att1: tipo1, att2: tipo2 ...) : tipo{
                     hi valor_retorno
                } imtheproblem
                \end{lstlisting}
                
                \subsubsection{Definición de los literales}
                Solo se perimitira el uso de comillas dobles para cualquier tipo de string (wonderstruck), pero ademas se incluyen los valores sparksfly y badblood para definir verdadero y falso respectivamente

                \subsubsection{Definición de los operadores}
                Se utilizaran los operadores +, -, *, / y el de porcentaje. Estos simbolos tendran la misma funcionalidad que cualquier otro lenguaje de programacion, ademas los parentesis mantienen su funcionalidad junto a los simbolos = y ==.
                
            \subsection{Expresiones regulares}
            \begin{table}[h]
\centering
\begin{tabular}{|c|c|}
\hline
\textbf{Token} & \textbf{Expresion regular} \\
\hline
identificador & [a − z][a − Z0 − 9]∗ \\
literal & ([a-Z]+|[0-9]+)+ \\
numeral & [0 − 9]+  || [0 − 9]+ . [0 − 9]+ \\
oper plus & numeral + numeral \\
oper mul & numeral * numeral \\
oper sus & numeral * numeral \\
oper division & numeral * numeral \\
men & > \\
may & < \\
plus & > \\
minus & < \\
times & > \\
divide & < \\
assign & = \\
compare & == \\
compare_or & || \\
compare_and & && \\

\hline
Comentarios & shake \\
Comentario bloque & shakeitoff expression shakeitoff \\
function & isme { (expression)* } imtheproblem  \\
return & hi \\
end function & imtheproblem \\
\hline
if & loverera (expression ) \{ (expression)*\} \\
else &  repera \{ (expression)*\} \\
elif & redera (expression ) \{ (expression)*\} \\
\hline
while & ooh ( expression ) \{expression*\} \\
break while & oohooh  \\
for & me (numeral) e (numeral) \{expression*\} \\
break for & ee  \\
\hline
boolean & meetyou (type\_bool) \\
type\_bool & (SparksFly | BadBlood) \\
\hline
double & thpage identificador \\
string & wonderstruck " (identificador) " \\
\hline
rigth\_p & ( \\
left\_p & ) \\
\hline
\end{tabular}
\caption{Tabla de tokens y expresiones regulares}
\label{tab:my_label}
\end{table}
            
            \subsection{Ejemplos de código}
                \subsubsection{Hola mundo}
                \begin{lstlisting}[caption={}][H]
                speaknow("Hola Mundo")
                \end{lstlisting}
                \subsubsection{Factorial iterativo}
                \begin{lstlisting}[caption={}][H]
                isme factorial1 (number: thpage) : thpage{
                    enchanted total = 0 : thpage
                    me 1 e number{
                        total *= e
                    }
                     hi total
                } imtheproblem
                \end{lstlisting}
                \subsubsection{Factorial recursivo}
                
                \begin{lstlisting}[caption={}][H]
                isme factorial2 (number: thpage) : thpage{
                    loverera number == 0 {
                        hi 1
                    } repera {
                        hi number * factorial2(number - 1)
                    }
                } imtheproblem
                \end{lstlisting}

	
	%\clearpage
	%\bibliographystyle{apalike}
	%\bibliographystyle{IEEEtranN}
	%\bibliography{bibliography}
		
	
\end{document}
